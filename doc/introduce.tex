\documentclass[12pt,a4paper]{article}
\usepackage[utf8x]{inputenc}
\usepackage[T1]{fontenc}
\usepackage{graphicx}
\usepackage{ucs}
\usepackage[polish]{babel}
\usepackage{geometry}

\title{Tanki\\
Dokumentacja wstępna }

\author{
Paweł Szewczyk
\and
Michał Bloch
}


\geometry{a4paper}

\begin{document}
\maketitle

\section{Zasady gry}
\emph{Tanki} to turowa gra artyleryjska, w której gracze mają za cel zniszczyć przeciwników przy użyciu kontrolowanych przez siebie czołgów.
\subsection{Przebieg rozgrywki}
Gracze naprzemiennie wykonują swoje tury.
W każdej turze gracz dokonuje jednego strzału jednym ze swoich czołgów, dobierając odpowiedni kąt i siłę. Na tor lotu wystrzelonego pocisku (lub pocisków) mogą mieć wpływ warunki panujące na planszy (siła i kierunek wiatru, grawitacja). Pocisk który trafia w czołg powoduje zmniejszenie jego \emph{punktów życia}. Zmniejszenie punktów życia do zera oznacza wyeliminowanie czołgu.
\subsection{Zakończenie gry}
Wygrywa gracz, który wyeliminuje wszystkie czołgi przeciwnika. Możliwy jest również remis, kiedy w jednej turze zostaną wyeliminowane wszystkie obecne na planszy czołgi.

\section{Przewidziana funkcjonalność}
\subsection{Podstawowe elementy}
Podstawowa wersja gry pozwala na rozegranie gry na podanych zasadach, również przez sieć. Składają się na to:
\begin{itemize}
\item celowanie
\item implementacja podstawowej fizyki pocisku (grawitacja, wiatr)
\item wykrywanie kolizji pocisku z czołgiem
\item zniszczenia czołgów, ich eliminacja z gry
\item komunikacja sieciowa z przeciwnikiem
\end{itemize}

\subsection{Dodatkowe elementy}
Dodatkowa funkcjonalność obejmuje szereg usprawnień i urozmaiceń rozgrywki:
\begin{itemize}
\item generowanie losowej planszy
\item destrukcja planszy
\item obsługa wielu rodzajów broni
\item możliwość poruszania się czołgiem
\item dodatkowe elementy fizyki pocisku (np. opór powietrza)
\item różne tryby rozgrywki
\end{itemize}
Dodawanie nowych funkcjonalności wiąże się często z potrzebą dodatkowej komunikacji sieciowej (np generowanie losowej planszy wymaga jej przesyłania).

\section{Podział pracy}

Michał Bloch:
\begin{itemize}
\item komunikacja sieciowa
\item interakcja z graczem
\item różne rodzaje broni i tryby rozgrywki
\end{itemize}
Paweł Szewczyk:
\begin{itemize}
\item rysowanie mapy i efektów graficznych
\item fizyka pocisku i otoczenia
\end{itemize}

\end{document}